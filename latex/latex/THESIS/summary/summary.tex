\chapter{Conclusion and Summary}

\label{ch:summary}

Artificial intelligence is the design and study of computer programs that react
flexibly and intelligently to a wide variety of situations. It has growing influence in new
computer related technologies and makes many complicated tasks possible. The
development of new hardware and techniques is fueling an ongo ing movement to build
computer systems that can understand and think in a cognitive way. While the potential
advantages of such systems is yet unknown, equally unknown are the potential pitfalls of
developing intelligent machinery.\\

Several technological leaders point to the course of history and their own
experiences in saying that artificially intelligent machines may soon become a reality.
These machines, if developed, may outlive and outgrow humanity on Earth. They may
forcefully take over the planet, or may not take it over at all. Human beings may even
learn to evolve into machines and reach a sort of immortality. In scientific outlooks, we
must prepare for what is possible, rather than what is certain. Engineers are best suited to
spot potential pitfalls of AI and other technologies, and should individually adhere to
stringent professional ethics in the practice of their art. But it is equally important that
ethical people with engineering education and experience become more intimately
involved in decision making and policy drafting processes through communication and
an expanded educational curriculum.


%\section{Project management}




%\section{Contributions and reflections}


