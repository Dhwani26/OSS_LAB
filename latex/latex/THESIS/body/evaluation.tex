\chapter{Recommendations}

%Explaining how your software was tested (using different datasets or in different environments), statistical evaluation of performance, results of user evaluation questionnaires, etc.

This paper considers artificial intelligence and its place in our future. The
predictions presented have multiple variations, but one clear underlying theme. Several
leading minds in the fields of computers and robotics believe that artificially intelligent
machines will be created. This chapter outlines recommendations for the engineering
community to foster the artificial intelligence movement in a safe and sustainable way.

\section{Programming For a Sustainable Future}
Ultimately, sustainability is our best ally. Without the future we have nowhere to
go (Poritt). It is therefore essential that each person and professional do his or her part to
build a sustainable future. That is not to say that progress should stop, only that care
should be taken to consider the conseque nces of technological development. We should
continue to develop, but not in a way that is harmful to humanity or to our planet. For
computer scientists, this may not seem like a difficult task. We may not realize that we
are capable of building a non-sustainable future. We may or may not have the power to
build technologies that can destroy our future. It is our responsibility as professionals not
to develop such technologies if we can help it. This, again, is not to say that we should
stop developing technologies altogether. Any technology, new or old, can be used
improperly, a problem as impossible to avoid as it is to predict.\\
In spite of the ethical points raised in this paper, we should not forget our most
obvious professional responsibility. In order to remain useful in this society, computer
scientists will continue to develop the technologies that society demands. This is the
reason that we are all engineers, to build technology. The public will continue to want new and exciting things. They will continue to need better interfaces and more complex
software systems. Building more intelligent software is the best way to meet these new
needs. Machines that are artificially intelligent, in some capacity, will continue to
become a necessity. More people will crowd onto our planet, needing more resources
delivered to them at a faster rate. Only technology will provide that.\\
On the other hand, we are required by our own professional ethics to protect the
public from that which may harm them. We can do nothing for society if we allow
technology to get beyond our control. The responsibility falls on us because we know
better than anyone what we are capable of creating, and we should know better than
anyone what those creations are capable of doing. This is true for any technology, not
just artificial intelligence. While we may not see the danger or potential in any of our
products, it is time to start thinking more seriously about our effect on the rest of the
world.\\

\subsection{Meeting Higher Levels of Responsibility}
My research throughout this project led me to believe that computer professionals,
as a community, lack a strong governing body. As we blaze ahead into the future and
create new, wonderful, and possibly dangerous things, we need guidance. While our
products affect as many people as those of any other industry, it remains true that there is
no FDA to test and regulate the production of our software, and no Bar Association to
keep us from practicing computer science in an unethical or unprofessional way. It
seems we need someone to make sure that we are doing the right thing. In hindsight,
however, these organizations are not the answer. They are not plausible. No one body could possibly keep watch over all computer programmers. We would be so bogged
down in ourselves that technologies would never come to fruition.\\
On one level, we must therefore practice individual self-governance. We must be
the ones to make sure we are doing the right thing. As engineers, we should be proud to
have a code of ethics, and hold it close to our hearts. But there is only so much we can
each be expected to do individually. Individual moral standards, while necessary, are not
sufficient. We have taken it upon ourselves to be technological leaders of our generation,
and are responsible for acting like leaders.\\
The growing influence of computer systems demands that computer scientists
become more active in the decision making process. It is too much responsibility for
every computer programmer to evaluate the moral justification of his or her project every
day. An individual programmer may not even know what puzzle his or her code will fit
into. Management hierarchies exist because history has shown that the evaluation of
moral, ethical, and logistical dimensions of a work product is in itself a full-time job.
Computer programmers producing their work pieces need to be confident that the
decisions handed down to them are trustworthy, safe, and ethical.\\
Yet, when safety budgets are recalculated, only engineers can fully understand
how much security is safe enough. Only engineers who are intimately involved with
multibillion-dollar space shuttles should ultimately decide whether ambient conditions
are safe for launch. It would seem that engineers should have a strong hand in
developing domestic and foreign policies, as advancing technology will continue to make
our world a smaller place. Ethical people with technological knowledge should also
pursue politics, management, and other policy drafting fields. That is how to ensure that developing technologies will be useful and safe. Not all engineers can be digging the
trenches.\\
AI will surely continue to develop in the future. In whatever methodology we
choose, it is our duty to be mindful of the future. As a basis we know that we must build
the right product, and build it as best we can. But we must also ensure that our
developments maintain a sustainable future.


