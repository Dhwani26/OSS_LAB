\chapter{A More Optimistic View}
Many prominent figures in modern technology believe that artificial intelligence
will become a reality in the not too distant future. Some also agree that intelligent
machines will succeed humanity. Unlike previously discussed dystopian views, however,
there are those who welcome the advance of intelligent machines. These futurists believe
that human beings will combine with robots or else foster them as our progeny. While
wildly technocratic, these views have their own basis in the current trends toward rapid
technological advance. This chapter will review the utopian futurist views of artificially
intelligent machines.
\section{Not Quite The Matrix: Intelligent Machines Still Succeed Humanity}
Technological development, it seems, is inevitable. Technology continues to be
driven by the needs and desires of society. Even the luddites draw arbitrary lines
between acceptable and unacceptable technology. Those who shun technology in theory
surely don’t survive by their bare hands alone. Technology is a tool, and there are those
who believe that proper use of technology as a tool can help make life itself more
fulfilling \cite{five}. By this token, it seems logical that continued development of
technologies can make life even more enjoyable in the future.\\
Certainly it is not the goal of computer scientists to develop software that will
destroy humanity. It is equally unlikely that engineers in the computing field believe that
they will develop an artificial life, only to shut it down and murder it. Computer
scientists build their products as a service. These engineers strive to build better
programs because they want to better serve those who use the programs. In many cases, Certainly it is not the goal of computer scientists to develop software that will
destroy humanity. It is equally unlikely that engineers in the computing field believe that
they will develop an artificial life, only to shut it down and murder it. Computer
scientists build their products as a service. These engineers strive to build better
programs because they want to better serve those who use the programs. In many cases, this means building more intelligent software. Sometimes it also means building
intelligence into a robot. But robots should not necessarily represent the locust plague.
There may perhaps be a scenario in which human beings and intelligent machines share
their existences.\\
Stanley Kubrick & Steven Spielburgs’s 2001 film, AI: The Artificial Intelligence
is a strong modern revision to Kubrick’s 1968 production of 2001: A Space Odyssey
(originally written by Arthur C. Clarke). Both feature the introduction of machines with
human-like intelligence. However, the newer film portrays robots as more cooperative
and aware of their own fallibility. This is a more human- like upgrade to the unruly and
overconfident HAL 9000. The Artificial Intelligence also paints for humans a more
disdainful and regrettably more probable attitude toward intelligent machines. Given the
creation of sufficiently intelligent machines, the movie shows how machines and people
might live together. It envisions robots in a largely subservient role, providing
continually greater service to their human counterparts, including even various emotional
services.\\
The end of Kubrick’s cinematic vision predicts that robots outlive humanity and
carry the torch of life on Earth when the environment becomes too inhospitable for
human life. Hans Moravec finds this to be an attractive and likely scenario for the future
of humanity and robotics. He believes that our human desire to propagate our species
will eventually manifest itself in a more metaphysical way, and that we will surrender
dominance of Earth in exchange for a sort of immortality \cite{seven}. The development
of robots with an intelligence of their own could present a way for humans to outlive themselves. It may be possible that our desire to live on would be satisfied by the living
of immortal machines that we foster with our own thoughts and teachings.\\
Kurzweil again outdoes Morvec by predicting that we will not allow robots to
succeed us. Instead, Kurzweil writes that we will join with machines and become a race
of cyborg humans . The change, he admits, will happen slowly and
gradually. As evidence to support his position, prosthetic devices are becoming more
commonplace and more technologically advanced. We have developed artificial legs and
arms, even artificial hearts. As we learn more about the human body, there is little to
stop us from emulating it in technology. Kurzweil believes that eventually we will even
have microcircuits in our brains. These microcircuits will be capable of
increased mathematical processing and memory storage. They will also, Kurzweil
purports, allow us to directly manipulate our own thoughts. By running a program on
these circuits, we can and will live in an increasingly virtual world that will be so real to
us that we can’t tell the difference from reality.\\
There are some points about Kurzweil’s vision that are appealing. It might be
nice, for instance, to directly stimulate our own joy. But human beings are not likely to
become cyborgs, at least not in the near term. Despite the continuing rush of technology,
people will not accept such a drastic mutation of our bodies. Computer programs
ultimately exist to serve society, and there will not be enough social support for a race of
half- humans. As dependent as we are on technology, there is still something sacred about
our bodies. Human beings do not want to be robots, and we might not really even want
to live forever.\\
Of course, there is no accounting for science. Arthur C. Clarke’s first law of
technology states,
\begin{center}
    \say{\textit{When a scientist states that something is possible, he is almost certainly right.\\
When he states that something is impossible, he is very probably wrong.}}
\end{center}
In other words, history has shown that technology is like an unstoppable train. Human
beings have learned to fly through sky and space, and travel to the greatest depths of the
ocean. Preparing for the unpredictable future is more about prospects and probabilities
than about certainties.\\
Solid testimony from some industry leaders supports the idea that we may soon be
living amongst artificially intelligent machines. The opinions of experts represented here
certainly do not guarantee the eventual creation of truly intelligent machinery, but we
must plan according to what may happen because we don’t know what will happen. The
eventual development of powerful artificial intelligence systems may or may not lead to a
maligned race of robots. Any outcome, however, will certainly carry serious
consequences for engineers and all other citizens. We must, therefore, be mindful
throughout our journey into the future of AI development, and be prepared for whatever
we find there.
%Containing a comprehensive description of the implementation of your software, including the language(s) and platform chosen, problems encountered, any changes made to the design as a result of the implementation, etc.