\begin{enumerate}
    As a scientific endeavour, machine learning grew out of the quest for artificial intelligence. Already in the early days of AI as an academic discipline, some researchers were interested in having machines learn from data. They attempted to approach the problem with various symbolic methods, as well as what were then termed \say{neural networks}; these were mostly perceptrons and other models that were later found to be reinventions of the generalized linear models of statistics.Probabilistic reasoning was also employed, especially in automated medical diagnosis.Machine learning, reorganized as a separate field, started to flourish in the 1990s. The field changed its goal from achieving artificial intelligence to tackling solvable problems of a practical nature. It shifted focus away from the symbolic approaches it had inherited from AI, and toward methods and models borrowed from statistics and probability theory.It also benefited from the increasing availability of digitized information, and the ability to distribute it via the Internet. 
    
    \subsection{RELATION TO DATA MINING}
\end{enumerate}